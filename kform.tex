\documentclass[dvipdfmx]{jsarticle}
\usepackage{penrosetensor}

\begin{document}

\section{微分形式の公式の表現}
\label{sec: k-form}

\subsection{前提知識・表記上の注意}

本節では
\begin{itemize}
    \item 微分形式の定義
    \item 完全反対称Levi-Civitaテンソル
    \item wedge積
    \item 外微分
    \item 内部積
    \item Hodge作用素
\end{itemize}
の知識を前提とする.

基本的に添字を使う場合はEinsteinの縮約記法に従って表す.
また, 微分記号$\partial_i$は括弧を使わない限り\textbf{常に直後の量のみを微分する}.
すなわち, $\partial_iA^jB^k=\partial_i(A^j)B^k$.

$n$次対称群を$\mathfrak{S}_n$で表し, $P\in\mathfrak{S}_n$の符号を$\sgn(P)$とする.
$k$階の完全反対称Levi-Civitaテンソルは以下で定義する.
\begin{align*}
    &
    \epsilon_{\mu_1\cdots\mu_k}
    \equiv
    \det\mqty(
        \delta_{\mu_1}^1 & \cdots & \delta_{\mu_k}^1
        \\
        \vdots & \ddots & \vdots
        \\
        \delta_{\mu_1}^k & \cdots & \delta_{\mu_k}^k
    )
    =
    \sum_{P\in\mathfrak{S}_n}\sgn(P)
    \delta_{\mu_1}^{P(1)}
    \cdots
    \delta_{\mu_k}^{P(k)},
    \\
    &
    \epsilon^{\mu_1\cdots\mu_k}
    \equiv
    \delta^{\mu_1\nu_1}\cdots\delta^{\mu_k\nu_k}
    \epsilon_{\nu_1\cdots\nu_k},
    \\
    &
    \epsilon_{\mu_1\cdots\mu_k}^{\nu_1\cdots\nu_k}
    \equiv
    \epsilon_{\mu_1\cdots\mu_k}
    \epsilon^{\nu_1\cdots\nu_k}
\end{align*}
ただし\ref{sec: k-form: differential form on riemann manifold}で扱うRiemann多様体の場合は
\begin{equation}
    \label{eq: k-form: contravariant Levi-Civita}
    \epsilon^{\mu_1\cdots\mu_k}
    \equiv
    g^{\mu_1\nu_1}\cdots g^{\mu_k\nu_k}
\end{equation}
とする.



\subsection{微分形式及び各種演算}

\subsubsection{k-form}

k-formの基底は座標基底の双対基底で
\begin{align*}
    \dd{x}^{\mu_1}\wedge\cdots\wedge\dd{x}^{\mu_k}
    =
    \sum_{P\in\mathfrak{S}_k}\sgn(P)\dd{x}^{\mu_{P(1)}}\otimes\cdots\otimes\dd{x}^{\mu_{P(k)}}
    =
    \epsilon^{\mu_1\cdots\mu_k}_{\mu_1'\cdots\mu_k'}\dd{x}^{\mu_1'}\otimes\cdots\otimes\dd{x}^{\mu_k'}
\end{align*}
と表せ, 成分と合わせると
\begin{align*}
    \omega
    \equiv
    \frac{1}{k!}
    \omega_{\mu_1\cdots\mu_k}\dd{x}^{\mu_1}\wedge\cdots\wedge\dd{x}^{\mu_k}
    =
    \frac{1}{k!}
    \omega_{\mu_1\cdots\mu_k}\epsilon^{\mu_1\cdots\mu_k}_{\mu_1'\cdots\mu_k'}
    \dd{x}^{\mu_1'}\otimes\cdots\otimes\dd{x}^{\mu_k'}
\end{align*}
である.

Levi-Civitaテンソルの表現をもとにk-formは以下のように表される.
\begin{equation*}
    \frac{1}{k!}
    \omega_{\mu_1\cdots\mu_k}
    =
    \frac{1}{k!}
    \vcenter{\hbox{\begin{tikzpicture}
    \coordinate(origin1)at(0,0);
    \draw[ultra thick](origin1)--++(1.5,0);
    \draw($(origin1)+(0,.25)$)rectangle++(1.5,.4);
    \node at($(origin1)+(.75,.45)$)[anchor=center]{$\omega$};
    \draw
    ($(origin1)+(.1,.25)$)--++(0,-.5)node[anchor=north](mu1){$\mu_1$}
    ++(.4,.5)--++(0,-.5)node[anchor=north](mu2){$\mu_2$}
    ++(.4,0)node[anchor=center](cdots){$\cdots$}
    ($(origin1)+(1.4,.25)$)--++(0,-.5)node[anchor=north](muk){$\mu_k$};
\end{tikzpicture}
}}
\end{equation*}
太線はLevi-Civitaテンソルを表し, $\omega$とラベリングされている長方形は係数を表す.

グラフ記法では基底を表すのが難しいので, 原則として$\dd{x}$などの記号は書かない.
上に開いている脚は座標基底$\partial_\mu$に, 下に開いている脚は双対基底$\dd{x}^\mu$に繋がっていると解釈する.


\subsubsection{Lie微分}

テンソルに対するLie微分は一般に
\begin{equation}
    \label{eq: Lie derivative}
    \begin{split}
        &
        \mathcal{L}_V\qty(
            t_{\nu_1\cdots\nu_k}^{\mu_1\cdots\mu_l}
            \partial_{\mu_1}\otimes\cdots\partial_{\mu_l}
            \otimes\dd{x}^{\nu_1}\otimes\cdots\otimes\dd{x}^{\nu_k}
        )
        \\
        &=
        V^\lambda\partial_\lambda t_{\nu_1\cdots\nu_k}^{\mu_1\cdots\mu_l}
        \partial_{\mu_1}\otimes\cdots\otimes\partial_{\mu_l}
        \otimes\dd{x}^{\nu_1}\otimes\cdots\otimes\dd{x}^{\nu_k}
        \\
        &\quad
        -
        t_{\nu_1\cdots\nu_k}^{\mu_1\cdots\mu_l}
        \partial_{\mu_1}V^\lambda\partial_\lambda\otimes\cdots\otimes\partial_{\mu_l}
        \otimes\dd{x}^{\nu_1}\otimes\cdots\otimes\dd{x}^{\nu_k}
        \\
        &\quad
        -\cdots
        \\
        &\quad
        -
        t_{\nu_1\cdots\nu_k}^{\mu_1\cdots\mu_l}
        \partial_{\mu_1}\otimes\cdots\otimes\partial_{\mu_l}V^\lambda\partial_\lambda
        \otimes\dd{x}^{\nu_1}\otimes\cdots\otimes\dd{x}^{\nu_k}
        \\
        &\quad
        +
        t_{\nu_1\cdots\nu_k}^{\mu_1\cdots\mu_l}
        \partial_{\mu_1}\otimes\cdots\otimes\partial_{\mu_l}
        \otimes \partial_\lambda V^{\nu_1}\dd{x}^{\lambda}\otimes\cdots\otimes\dd{x}^{\nu_k}
        \\
        &\quad
        +\cdots
        \\
        &\quad
        +
        t_{\nu_1\cdots\nu_k}^{\mu_1\cdots\mu_l}
        \partial_{\mu_1}\otimes\cdots\otimes\partial_{\mu_l}
        \otimes\dd{x}^{\nu_1}\otimes\cdots\otimes\partial_\lambda V^{\nu_k}\dd{x}^{\lambda}
    \end{split}
\end{equation}
で表される.

これをPenroseのグラフ記法で表すと以下の通り.
\begin{equation*}
    \vcenter{\hbox{\begin{tikzpicture}
    \coordinate(origin1)at(0,-.2);
    \coordinate(origin2)at(2.8,-.2);
    \coordinate(origin3)at(3,-2.);
    \coordinate(origin4)at(3,-3.7);
    \node at(-.5,0)[anchor=center]{$\mathcal{L}_V$};
    \node at(2.,0)[anchor=center]{$=$};
    \node at(2.5,-2)[anchor=center]{$-$};
    \node at(5.2,-2)[anchor=center]{$-\cdots$};
    \node at(2.5,-3.7)[anchor=center]{$+$};
    \node at(5.2,-3.7)[anchor=center]{$+\cdots$};

    % LHS
    \draw
    (origin1)rectangle++(1.5,.4);
    \node at($(origin1)+(.75,.2)$)[anchor=center]{$t$};
    \draw
    ($(origin1)+(.1,.4)$)--++(0,.25)
    ++(.4,-.25)--++(0,.25)
    ++(.4,0)node[anchor=center]{$\cdots$}
    ($(origin1)+(1.5,.4)-(.1,0)$)--++(0,.25)
    ($(origin1)+(.1,0)$)--++(0,-.25)
    ++(.4,.25)--++(0,-.25)
    ++(.4,0)node[anchor=center]{$\cdots$}
    ($(origin1)+(1.5,0)-(.1,0)$)--++(0,-.25)
    ;

    % RHS diff normally
    \draw
    (origin2)rectangle++(1.5,.4);
    \node at($(origin2)+(.75,.2)$)[anchor=center]{$t$};
    \draw
    ($(origin2)+(.1,.4)$)--++(0,.25)
    ++(.4,-.25)--++(0,.25)
    ++(.4,0)node[anchor=center]{$\cdots$}
    ($(origin2)+(1.5,.4)-(.1,0)$)--++(0,.25)
    ($(origin2)+(.1,0)$)--++(0,-.25)
    ++(.4,.25)--++(0,-.25)
    ++(.4,0)node[anchor=center]{$\cdots$}
    ($(origin2)+(1.5,0)-(.1,0)$)--++(0,-.25)
    ($(origin2)+(.75,.2)$)circle[x radius=1,y radius=.35]
    ++(-.95,-.1)--++(0,-.25)node[anchor=north, draw, rectangle]{$V$}
    ;

    % RHS on vector
    \draw(origin3)rectangle++(1.5,-.4);
    \node at($(origin3)+(.75,-.2)$)[anchor=center]{$t$};
    \node at($(origin3)+(.1,.5)$)[anchor=center, draw, rectangle](VonVector){$V$};
    \draw
    (VonVector.north)--++(0,.25)
    (VonVector.center)circle[radius=.35]
    ++(0,-.35)--($(origin3)+(.1,0)$)
    ($(origin3)+(.5,0)$)--++(0,.25)
    ++(.4,0)node[anchor=center]{$\cdots$}
    ($(origin3)+(.1,-.4)$)--++(0,-.25)
    ++(.5,.25)--++(0,-.25)
    ++(.4,0)node[anchor=center]{$\cdots$}
    ;

    % RHS on covector
    \draw(origin4)rectangle++(1.5,.4);
    \node at($(origin4)+(.75,.2)$)[anchor=center]{$t$};
    \node at($(origin4)+(.1,-.5)$)[anchor=center, draw, rectangle](VonCovector){$V$};
    \draw
    (VonCovector.north)--($(origin4)+(.1,0)$)
    (VonCovector.center)circle[radius=.35]
    ++(0,-.35)--++(0,-.25)
    ($(origin4)+(.5,0)$)--++(0,-.25)
    ++(.4,0)node[anchor=center]{$\cdots$}
    ($(origin4)+(.1,.4)$)--++(0,.25)
    ++(.5,-.25)--++(0,.25)
    ++(.4,0)node[anchor=center]{$\cdots$}
    ;
\end{tikzpicture}
}}
\end{equation*}
\eqref{eq: Lie derivative}の表示でも使える直観的な作用素の付き方の判別法を紹介しよう.
微分作用素$V=V^\mu\partial_\mu$はLeibnitz ruleに従って各々の要素を微分していく.
成分$t_{\nu_1\cdots\nu_k}^{\mu_1\cdots\mu_l}$を微分する際は特に何も考えず丸で囲って脚を$V$に繋げれば良い.
座標基底$\partial_\mu$に作用する際は, 成分の四角と基底から伸びる脚の間に$V^\mu$と$\partial_\nu$が差し込まれる.
\begin{equation*}
    \vcenter{\hbox{\begin{tikzpicture}
    \coordinate(origin1)at(0,0);
    \node at(0,-.2)[anchor=center]{$t$};
    \node at(0,.5)[anchor=center, draw, rectangle](V){$V$};
    \draw
    ($(origin1)+(-.6,-.4)$)rectangle++(1.2,.4)
    (V.center)circle[radius=.35]
    ++(0,-.35)--(origin1)
    (V.north)--++(0,.25)
    ;
\end{tikzpicture}
}}
    \qquad
    \vcenter{\hbox{\begin{tikzpicture}
    \coordinate(origin1)at(0,0);
    \node at(0,.2)[anchor=center]{$t$};
    \node at(0,-.5)[anchor=center, draw, rectangle](V){$V$};
    \draw
    ($(origin1)+(-.6,0)$)rectangle++(1.2,.4)
    (V.center)circle[radius=.35]
    ++(0,-.35)--++(0,-.25)
    (V.north)--(origin1)
    ;
\end{tikzpicture}
}}
\end{equation*}
$V^\mu$を反変で, $\partial_\nu$を共変で差し込める形状は左図の形のみである.
同様にして双対基底$\dd{x}^\mu$に作用する場合を考えると, 右の場合だけが許される.


\subsubsection{wedge積}

外積(wedge積)は
$\xi\in\Omega^k(M), \eta\in\Omega^l(M)$に対して
\begin{align*}
    (\xi\wedge\eta)(V_1,\cdots,V_{k+l})
    =
    \frac{1}{k!l!}
    \sum_{P\in\mathfrak{S}_{k+l}}\sgn(P)\xi(V_{P(1)},\cdots V_{P(k)})\eta(V_{P(k+1)},\cdots V_{P(k+l)})
\end{align*}
で定義されるが,
成分表示してベクトルを除くと
\begin{equation*}
    \begin{split}
        &
        \qty(
            \frac{1}{k!}
            \xi_{\mu_1\cdots\mu_k}
            \dd{x}^{\mu_1}
            \wedge\cdots\wedge
            \dd{x}^{\mu_k}
        )
        \wedge
        \qty(
            \frac{1}{l!}
            \eta_{\mu_{k+1}\cdots\mu_{k+l}}
            \dd{x}^{\mu_{k+1}}
            \wedge\cdots\wedge
            \dd{x}^{\mu_{k+l}}
        )
        \\
        &=
        \frac{1}{k!l!}
        \xi_{\mu_1\cdots\mu_k}
        \eta_{\mu_{k+1}\cdots\mu_{k+l}}
        \dd{x}^{\mu_1}
        \wedge\cdots\wedge
        \dd{x}^{\mu_{k+l}}
        \\
        &=
        \frac{1}{k!l!}
        \xi_{\mu_1\cdots\mu_k}
        \eta_{\mu_{k+1}\cdots\mu_{k+l}}
        \epsilon^{\mu_1\cdots\mu_k\mu_{k+1}\cdots\mu_{k+l}}_{\mu_1'\cdots\mu_k'\mu_{k+1}'\cdots\mu_{k+l}}
        \dd{x}^{\mu_1'}
        \otimes\cdots\otimes
        \dd{x}^{\mu_{k+l}'}
    \end{split}
\end{equation*}
となる.

グラフ記法では, 2つの微分形式それぞれを貫くLevi-Civita記号の太線をつなげることでwedge積を表す.
\begin{equation*}
    % xi
    \frac{1}{k!}
    \vcenter{\hbox{
        \begin{tikzpicture}
            \coordinate(origin)at(0,0);
            \draw[ultra thick](origin)--++(1.5,0);
            \draw($(origin)+(0,.25)$)rectangle++(1.5,.4);
            \node at($(origin)+(.75,.45)$)[anchor=center]{$\xi$};
            \draw
                ($(origin)+(.1,.25)$)--++(0,-.5)node[anchor=north]{$\mu_1$}
                ++(.4,.5)--++(0,-.5)node[anchor=north]{$\mu_2$}
                ++(.4,0)node[anchor=center](cdots){$\cdots$}
                ($(origin)+(1.4,.25)$)--++(0,-.5)node[anchor=north]{$\mu_k$}
            ;
        \end{tikzpicture}
    }}
    \wedge
    % eta
    \frac{1}{l!}
    \vcenter{\hbox{
        \begin{tikzpicture}
            \coordinate(origin)at(0,0);
            \draw[ultra thick](origin)--++(1.5,0);
            \draw($(origin)+(0,.25)$)rectangle++(1.5,.4);
            \node at($(origin)+(.75,.45)$)[anchor=center]{$\eta$};
            \draw
                ($(origin)+(.1,.25)$)--++(0,-.5)node[anchor=north]{$\mu_{k+1}$}
                ++(.4,0)node[anchor=center](cdots){$\cdots$}
                ($(origin)+(1.4,.25)$)--++(0,-.5)node[anchor=north]{$\mu_{k+l}$}
            ;
        \end{tikzpicture}
    }}
    =
    % xi wedge eta
    \frac{1}{k!l!}
    \vcenter{\hbox{
        \begin{tikzpicture}
            \coordinate(origin)at(0,0);
            \draw[ultra thick](origin)--++(3.6,0);
            \draw
                ($(origin)+(0,.25)$)rectangle++(1.5,.4)
                ($(origin)+(.1,.25)$)--++(0,-.5)node[anchor=north]{$\mu_{1}$}
                ++(.4,0)node[anchor=center]{$\cdots$}
                ($(origin)+(1.4,.25)$)--++(0,-.5)node[anchor=north]{$\mu_{k}$};
                \node at($(origin)+(.75,.45)$)[anchor=center]{$\xi$}
            ;
            \draw
                ($(origin)+(2.1,.25)$)rectangle++(1.5,.4)
                ($(origin)+(2.2,.25)$)--++(0,-.5)node[anchor=north]{$\mu_{k+1}$}
                ++(.5,0)node[anchor=center]{$\cdots$}
                ($(origin)+(3.5,.25)$)--++(0,-.5)node[anchor=north]{$\mu_{k+l}$};
                \node at($(origin)+(2.8,.45)$)[anchor=center]{$\eta$}
            ;
        \end{tikzpicture}
    }}
\end{equation*}


\subsubsection{外微分}

外微分は
\begin{equation*}
    \label{eq: d omega}
    \begin{split}
        \dd{\qty(
            \frac{1}{k!}
            \omega_{\mu_1\cdots\mu_k}
            \dd{x}^{\mu_1}
            \wedge\cdots\wedge
            \dd{x}^{\mu_k}
        )}
        &=
        \frac{1}{k!}
        \partial_{\mu_0}\omega_{\mu_1\cdots\mu_k}
        \dd{x}^{\mu_0}\wedge\dd{x}^{\mu_1}\wedge\cdots\wedge\dd{x}^{\mu_k}
        \\
        &=
        \frac{1}{k!}
        \partial_{\mu_0}\omega_{\mu_1\cdots\mu_k}
        \epsilon^{\mu_0\mu_1\cdots\mu_k}_{\mu_0'\mu_1'\cdots\mu_k'}
        \dd{x}^{\mu_0'}\otimes\cdots\otimes\dd{x}^{\mu_k'}
    \end{split}
\end{equation*}
で表される.

グラフ記法では係数を表す四角を微分記号の丸で囲い, 出した脚を太線の先頭に差し込む.
\begin{equation*}
    \dd{\qty(
        \frac{1}{k!}
        \vcenter{\hbox{
            \begin{tikzpicture}
                \coordinate(origin)at(0,0);
                \draw[ultra thick](origin1)--++(1.5,0);
                \draw($(origin)+(0,.25)$)rectangle++(1.5,.4);
                \node at($(origin)+(.75,.45)$)[anchor=center]{$\omega$};
                \draw
                    ($(origin)+(.1,.25)$)--++(0,-.5)node[anchor=north](mu1){$\mu_1$}
                    ++(.4,.5)--++(0,-.5)node[anchor=north](mu2){$\mu_2$}
                    ++(.4,0)node[anchor=center](cdots){$\cdots$}
                    ($(origin)+(1.4,.25)$)--++(0,-.5)node[anchor=north](muk){$\mu_k$}
                ;
            \end{tikzpicture}
        }}
    )}
    =
    % RHS
    \frac{1}{k!}
    \vcenter{\hbox{
        \begin{tikzpicture}
            \coordinate(origin)at(0,0);
            \draw[ultra thick](origin)--++(1.8,0);
            \draw($(origin)+(.3,.25)$)rectangle++(1.5,.4);
            \node at($(origin)+(1.05,.45)$)[anchor=center]{$\omega$};
            \draw
                ($(origin)+(.4,.25)$)--++(0,-.5)node[anchor=north](mu1){$\mu_1$}
                ++(.4,.5)--++(0,-.5)node[anchor=north](mu2){$\mu_2$}
                ++(.4,0)node[anchor=center](cdots){$\cdots$}
                ($(origin)+(1.7,.25)$)--++(0,-.5)node[anchor=north](muk){$\mu_k$};
                \draw
                ($(origin)+(1.05,.45)$)circle[x radius=1.,y radius=.35]
                ($(origin)+(.05,.4)$)--++(0,-.65)node[anchor=north]{$\mu_0$}
            ;
        \end{tikzpicture}
    }}
\end{equation*}


\subsubsection{内部積}
\label{sec: interior}

内部積は
\begin{align*}
    \iota_V\omega(V_1\cdots V_{k-1})
    =
    \omega(V,V_1\cdots V_{k-1})
\end{align*}
で定義されるが, これも成分表示によって
\begin{align*}
    \label{eq: interior product}
    \iota_V\qty(
        \frac{1}{k!}
        \omega_{\mu_1\cdots\mu_k}
        \dd{x}^{\mu_1}
        \wedge\cdots\wedge
        \dd{x}^{\mu_k}
    )
    =
    \frac{1}{k!}
    \omega_{\mu_1\cdots\mu_k}
    \epsilon_{\mu_1'\cdots\mu_k'}^{\mu_1\cdots\mu_k}
    V^{\mu_1'}\dd{x}^{\mu_2'}
    \otimes\cdots\otimes
    \dd{x}^{\mu_k'}
\end{align*}
となる.

グラフ記法で表すと, 先頭の脚を$V$で潰す形になる.
\begin{equation*}
    % LHS
    \iota_V
    \qty(
        \frac{1}{k!}
        \vcenter{\hbox{
            \begin{tikzpicture}
                \coordinate(origin)at(0,0);
                \draw[ultra thick](origin)--++(1.5,0);
                \draw($(origin)+(0,.25)$)rectangle++(1.5,.4);
                \node at($(origin)+(.75,.45)$)[anchor=center]{$\omega$};
                \draw
                    ($(origin)+(.1,.25)$)--++(0,-.5)node[anchor=north](mu1){$\mu_1$}
                    ++(.4,.5)--++(0,-.5)node[anchor=north](mu2){$\mu_2$}
                    ++(.4,0)node[anchor=center](cdots){$\cdots$}
                    ($(origin)+(1.4,.25)$)--++(0,-.5)node[anchor=north](muk){$\mu_k$}
                ;
            \end{tikzpicture}
        }}
    )
    =
    \frac{1}{k!}
    \vcenter{\hbox{
        \begin{tikzpicture}
            \coordinate(origin)at(0,0);
            \draw[ultra thick](origin)--++(1.5,0);
            \draw($(origin)+(0,.25)$)rectangle++(1.5,.4);
            \node at($(origin)+(.75,.45)$)[anchor=center]{$\omega$};
            \draw
                ($(origin)+(.1,.25)$)--++(0,-.5)node[anchor=north,draw,rectangle]{$V$}
                ++(.5,.5)--++(0,-.5)node[anchor=north](mu2){$\mu_2$}
                ++(.4,0)node[anchor=center](cdots){$\cdots$}
                ($(origin)+(1.4,.25)$)--++(0,-.5)node[anchor=north](muk){$\mu_k$}
            ;
        \end{tikzpicture}
    }}
\end{equation*}



\subsection{微分形式の公式}
\label{sec: k-form: formulae of k-form}

上記の表記を応用して微分形式の公式を直観的に導出する.
なお以下の導出はEinsteinの縮約記法を使っても, グラフ記法の流れをそのまま追うことで証明できる.

\subsubsection{Poincaréの補題 $\dd{}^2=0$}

微分順序の交換と反対称性を用いる.
\begin{equation*}
    \frac{1}{k!}
    \vcenter{\hbox{
        \begin{tikzpicture}
            \draw[ultra thick]($(origin1)+(-.4,0)$)--++(2.5,0);
            \draw($(origin1)+(.3,.25)$)rectangle++(1.5,.4);
            \node at($(origin1)+(1.05,.45)$)[anchor=center]{$\omega$};
            \draw
            ($(origin1)+(.4,.25)$)--++(0,-.5)
            ($(origin1)+(1.7,.25)$)--++(0,-.5);
            \draw
            ($(origin1)+(1.05,.45)$)circle[x radius=1.,y radius=.35]
            ($(origin1)+(1.05,.45)$)circle[x radius=1.1,y radius=.4]
            ++(-1.1,0)-|++(-.3,-.7)
            ($(origin1)+(.05,.4)$)--++(0,-.65)
            ;
        \end{tikzpicture}
    }}
    =
    \frac{1}{k!}
    \vcenter{\hbox{
        \begin{tikzpicture}
            \draw[ultra thick]($(origin2)+(-.4,0)$)--++(2.5,0);
            \draw($(origin2)+(.3,.25)$)rectangle++(1.5,.4);
            \node at($(origin2)+(1.05,.45)$)[anchor=center]{$\omega$};
            \draw
            ($(origin2)+(.4,.25)$)--++(0,-.5)
            ($(origin2)+(1.7,.25)$)--++(0,-.5);
            \draw
            ($(origin2)+(1.05,.45)$)circle[x radius=1.,y radius=.35]
            ++(-1,0)-|++(-.35,-.7)
            ($(origin2)+(1.05,.45)$)circle[x radius=1.1,y radius=.4]
            ($(origin2)+(.05,.3)$)--++(0,-.525)
            ;
        \end{tikzpicture}
    }}
    =
    -\frac{1}{k!}
    \vcenter{\hbox{
        \begin{tikzpicture}
            \draw[ultra thick]($(origin3)+(-.4,0)$)--++(2.5,0);
            \draw($(origin3)+(.3,.25)$)rectangle++(1.5,.4);
            \node at($(origin3)+(1.05,.45)$)[anchor=center]{$\omega$};
            \draw
            ($(origin3)+(.4,.25)$)--++(0,-.5)
            ($(origin3)+(1.7,.25)$)--++(0,-.5);
            \draw
            ($(origin3)+(1.05,.45)$)circle[x radius=1.,y radius=.35]
            ($(origin3)+(1.05,.45)$)circle[x radius=1.1,y radius=.4]
            ++(-1.1,0)-|++(-.3,-.7)
            ($(origin3)+(.05,.4)$)--++(0,-.65)
            ;
        \end{tikzpicture}
    }}
\end{equation*}
左辺と右辺は符号だけが違うので, $0$のみが許される.


\subsubsection{内部積の別表現}
\label{sec: interior 2}

内部積$\iota_V\omega$を表す方法として, ベクトル$V$で潰した$\omega$の先頭の脚をLevi-Civitaテンソルから外す表し方がある.
\begin{align*}
    \iota_V\omega
    &=
    k\frac{1}{k!}
    \omega_{\mu_1\cdots\mu_k}\epsilon^{\mu_2\cdots\mu_k}_{\mu_2'\cdots\mu_k'}
    V^{\mu_1'}\dd{x}^{\mu_2'}\otimes\cdots\otimes\dd{x}^{\mu_k'}
\end{align*}
Levi-Civitaテンソルの添字が$\mu_2, \mu_2'$から始まることに注意.

これを証明するにはまず$V$がつながった脚を反対称テンソルから外す.
\begin{equation*}
    \begin{split}
        \frac{1}{k!}
        \vcenter{\hbox{
            \begin{tikzpicture}
                \coordinate(origin1)at(0,0);
                \draw[ultra thick](origin1)--++(1.5,0);
                \draw($(origin1)+(0,.25)$)rectangle++(1.5,.4);
                \node at($(origin1)+(.75,.45)$)[anchor=center]{$\omega$};
                \draw
                ($(origin1)+(.1,.25)$)--++(0,-.5)node[anchor=north,draw,rectangle]{$V$}
                ++(.5,.5)--++(0,-.5)node[anchor=north](mu2){$\mu_2$}
                ++(.4,0)node[anchor=center](cdots){$\cdots$}
                ($(origin1)+(1.4,.25)$)--++(0,-.5)node[anchor=north](muk){$\mu_k$};
            \end{tikzpicture}
        }}
        &=
        \frac{1}{k!}
        \vcenter{\hbox{
            \begin{tikzpicture}
                \coordinate(origin2)at(2.5,0);
                \draw[ultra thick]($(origin2)+(.5,0)$)--++(1.,0);
                \draw($(origin2)+(0,.25)$)rectangle++(1.5,.4);
                \node at($(origin2)+(.75,.45)$)[anchor=center]{$\omega$};
                \draw
                ($(origin2)+(.1,.25)$)--++(0,-.5)node[anchor=north,draw,rectangle]{$V$}
                ++(.5,.5)--++(0,-.5)node[anchor=north](mu2){$\mu_2$}
                ++(.4,0)node[anchor=center](cdots){$\cdots$}
                ($(origin2)+(1.4,.25)$)--++(0,-.5)node[anchor=north](muk){$\mu_k$};
            \end{tikzpicture}
        }}
        -
        \frac{1}{k!}
        \vcenter{\hbox{
            \begin{tikzpicture}
                \coordinate(origin3)at(5.,0);
                \draw[ultra thick]($(origin3)+(.5,0)$)--++(1.,0);
                \draw($(origin3)+(0,.25)$)rectangle++(1.5,.4);
                \node at($(origin3)+(.75,.45)$)[anchor=center]{$\omega$};
                \draw
                ($(origin3)+(.1,.25)$)--++(.5,-.25)--++(0,-.25)node[anchor=north](mu2){$\mu_2$}
                ++(0,.5)--++(-.5,-.25)--++(0,-.25)node[anchor=north,draw,rectangle]{$V$}
                ++(1.,0)node[anchor=center](cdots){$\cdots$}
                ($(origin3)+(1.4,.25)$)--++(0,-.5)node[anchor=north](muk){$\mu_k$};
            \end{tikzpicture}
        }}
        +
        \frac{1}{k!}
        \vcenter{\hbox{
            \begin{tikzpicture}
                \coordinate(origin4)at(7.5,0);
                \draw[ultra thick]($(origin4)+(.5,0)$)--++(1.,0);
                \draw($(origin4)+(0,.25)$)rectangle++(1.5,.4);
                \node at($(origin4)+(.75,.45)$)[anchor=center]{$\omega$};
                \draw
                ($(origin4)+(.1,.25)$)--++(.5,-.25)--++(0,-.25)node[anchor=north](mu2){$\mu_2$}
                ++(0,.5)--++(.5,-.25)--++(0,-.25)node[anchor=north]{$\mu_3$}
                ++(0,.5)--++(-1,-.25)--++(0,-.25)node[anchor=north,draw,rectangle]{$V$}
                ++(1.5,0)node[anchor=center](cdots){$\cdots$}
                ;
            \end{tikzpicture}
        }}
        -
        \cdots
        \\
        &=
        \frac{k}{k!}
        \vcenter{\hbox{
            \begin{tikzpicture}
                \coordinate(origin5)at(3,-1.5);
                \draw[ultra thick]($(origin5)+(.5,0)$)--++(1.,0);
                \draw($(origin5)+(0,.25)$)rectangle++(1.5,.4);
                \node at($(origin5)+(.75,.45)$)[anchor=center]{$\omega$};
                \draw
                ($(origin5)+(.1,.25)$)--++(0,-.5)node[anchor=north,draw,rectangle]{$V$}
                ++(.5,.5)--++(0,-.5)node[anchor=north](mu2){$\mu_2$}
                ++(.4,0)node[anchor=center](cdots){$\cdots$}
                ($(origin5)+(1.4,.25)$)--++(0,-.5)node[anchor=north](muk){$\mu_k$};
            \end{tikzpicture}
        }}
    \end{split}
\end{equation*}
$\omega_{\mu_1\cdots\mu_k}$が添字に対して反対称であることに注意すると, 上図1段目右辺にて正号がつくものは偶置換で, 負号がつくものは奇置換で第1項に戻るので, 各項全て同じ値となる.
全部で$k$項あるので2段目を得る.


\subsubsection{Cartanの公式 $(\dd{\iota_V}+\iota_V\dd{})\omega=\mathcal{L}_V\omega$}

まずは左辺を書き出してみよう.
% 第1項について, \ref{sec: interior}で取り上げた内部積の表式にそのまま外微分を作用させようとすると, 基底の反対称テンソルにつながっている$V$が干渉してうまく微分ができない.
% そこで反対称テンソルから成分を分離する必要に迫られる.
% 以上の理由から,
第1項の内部積は\ref{sec: interior 2}を使うのが良い.
\begin{equation}
    \label{eqfig: kform: cartan1}
    \begin{split}
        (\dd{\iota_V}+\iota_V\dd{})
        \frac{1}{k!}
        \vcenter{\hbox{\begin{tikzpicture}
    \coordinate(origin1)at(0,0);
    \draw[ultra thick](origin1)--++(1.5,0);
    \draw($(origin1)+(0,.25)$)rectangle++(1.5,.4);
    \node at($(origin1)+(.75,.45)$)[anchor=center]{$\omega$};
    \draw
    ($(origin1)+(.1,.25)$)--++(0,-.5)node[anchor=north](mu1){$\mu_1$}
    ++(.4,.5)--++(0,-.5)node[anchor=north](mu2){$\mu_2$}
    ++(.4,0)node[anchor=center](cdots){$\cdots$}
    ($(origin1)+(1.4,.25)$)--++(0,-.5)node[anchor=north](muk){$\mu_k$};
\end{tikzpicture}
}}
        &=
        \dd{\frac{k}{k!}}
        \vcenter{\hbox{
            \begin{tikzpicture}
                \coordinate(origin2)at(0,0);
                \draw[ultra thick]($(origin2)+(.5,0)$)--++(1.,0);
                \draw($(origin2)+(0,.25)$)rectangle++(1.5,.4);
                \node at($(origin2)+(.75,.45)$)[anchor=center]{$\omega$};
                \draw
                ($(origin2)+(.1,.25)$)--++(0,-.5)node[anchor=north,draw,rectangle]{$V$}
                ++(.5,.5)--++(0,-.5)node[anchor=north](mu2){$\mu_2$}
                ++(.4,0)node[anchor=center](cdots){$\cdots$}
                ($(origin2)+(1.4,.25)$)--++(0,-.5)node[anchor=north](muk){$\mu_k$};
            \end{tikzpicture}
        }}
        +
        \iota_V
        \frac{1}{k!}
        \vcenter{\hbox{
            \begin{tikzpicture}
                \coordinate(origin3)at(6.,0);
                \draw[ultra thick](origin3)--++(1.8,0);
                \draw($(origin3)+(.3,.25)$)rectangle++(1.5,.4);
                \node at($(origin3)+(1.05,.45)$)[anchor=center]{$\omega$};
                \draw
                ($(origin3)+(.4,.25)$)--++(0,-.5)node[anchor=north](mu1){$\mu_1$}
                ++(.4,.5)--++(0,-.5)node[anchor=north](mu2){$\mu_2$}
                ++(.4,0)node[anchor=center](cdots){$\cdots$}
                ($(origin3)+(1.7,.25)$)--++(0,-.5)node[anchor=north](muk){$\mu_k$};
                \draw
                ($(origin3)+(1.05,.45)$)circle[x radius=1.,y radius=.35]
                ($(origin3)+(.05,.4)$)--++(0,-.65)node[anchor=north]{$\mu_0$}
                ;
            \end{tikzpicture}
        }}
        \\
        &=
        \frac{k}{k!}
        \vcenter{\hbox{
            \begin{tikzpicture}
                \coordinate(origin5)at(5.5,-2.5);
                \draw[ultra thick]($(origin4)+(.0,0)$)--++(1.5,0);
                \draw($(origin4)+(0,.75)$)rectangle++(1.5,.4);
                \node at($(origin4)+(.75,.95)$)[anchor=center]{$\omega$};
                \draw
                ($(origin4)+(.1,.75)$)--++(0,-.1)node[anchor=north,draw,rectangle]{$V$}
                ++(.5,.1)--++(0,-1)node[anchor=north](mu2){$\mu_2$}
                ++(.4,0)node[anchor=center](cdots){$\cdots$}
                ($(origin4)+(1.4,.75)$)--++(0,-1)node[anchor=north](muk){$\mu_k$}
                ($(origin4)+(.75,.7)$)circle[x radius=1.3,y radius=.6]
                ++(-.45,-.55)--($(origin4)+(.1,0)$)--($(origin4)+(.1,-.25)$)node[anchor=north]{$\mu_1$}
                ;
            \end{tikzpicture}
        }}
        +
        \frac{1}{k!}
        \vcenter{\hbox{
            \begin{tikzpicture}
                \coordinate(origin5)at(5.5,-2.5);
                \draw[ultra thick](origin5)--++(1.8,0);
                \draw($(origin5)+(.3,.25)$)rectangle++(1.5,.4);
                \node at($(origin5)+(1.05,.45)$)[anchor=center]{$\omega$};
                \draw
                ($(origin5)+(.5,.25)$)--++(0,-.5)node[anchor=north](mu1){$\mu_1$}
                ++(.4,.5)--++(0,-.5)node[anchor=north](mu2){$\mu_2$}
                ++(.4,0)node[anchor=center](cdots){$\cdots$}
                ($(origin5)+(1.7,.25)$)--++(0,-.5)node[anchor=north](muk){$\mu_k$};
                \draw
                ($(origin5)+(1.05,.45)$)circle[x radius=1.,y radius=.35]
                ($(origin5)+(.05,.4)$)--++(0,-.65)node[anchor=north, draw, rectangle]{$V$}
                ;
            \end{tikzpicture}
        }}
    \end{split}
\end{equation}
第1項は微分をLeibnitz ruleによって分解する.
\begin{equation*}
    \frac{k}{k!}
    \vcenter{\hbox{
        \begin{tikzpicture}
            \coordinate(origin1)at(-.5,0);
            \draw[ultra thick]($(origin1)+(.0,0)$)--++(1.5,0);
            \draw($(origin1)+(0,.75)$)rectangle++(1.5,.4);
            \node at($(origin1)+(.75,.95)$)[anchor=center]{$\omega$};
            \draw
            ($(origin1)+(.1,.75)$)--++(0,-.1)node[anchor=north,draw,rectangle]{$V$}
            ++(.5,.1)--++(0,-1)
            ++(.4,0)node[anchor=center](cdots){$\cdots$}
            ($(origin1)+(1.4,.75)$)--++(0,-1)
            ($(origin1)+(.75,.7)$)circle[x radius=1.3,y radius=.6]
            ++(-.45,-.55)--($(origin1)+(.1,0)$)--($(origin1)+(.1,-.25)$)
            ;
        \end{tikzpicture}
    }}
    =
    \frac{k}{k!}
    \vcenter{\hbox{
        \begin{tikzpicture}
            \coordinate(origin2)at(3,0);
            \draw[ultra thick]($(origin2)+(.0,0)$)--++(1.5,0);
            \draw($(origin2)+(0,.75)$)rectangle++(1.5,.4);
            \node at($(origin2)+(.75,.95)$)[anchor=center]{$\omega$};
            \draw
            ($(origin2)+(.1,.75)$)--++(0,-.1)node[anchor=north,draw,rectangle](V){$V$}
            ++(.5,.1)--++(0,-1)
            ++(.4,0)node[anchor=center](cdots){$\cdots$}
            ($(origin2)+(1.4,.75)$)--++(0,-1)
            (V)circle[radius=.3]
            ++(0,-.3)--($(origin2)+(.1,0)$)--++(0,-.25)
            ;
        \end{tikzpicture}
    }}
    +
    \frac{k}{k!}
    \vcenter{\hbox{
        \begin{tikzpicture}
            \coordinate(origin3)at(6.,0);
            \draw[ultra thick]($(origin3)+(.0,0)$)--++(1.5,0);
            \draw($(origin3)+(0,.75)$)rectangle++(1.5,.4);
            \node at($(origin3)+(.75,.95)$)[anchor=center](omega){$\omega$};
            \draw
            ($(origin3)+(.1,.75)$)--++(0,-.1)node[anchor=north,draw,rectangle]{$V$}
            ++(.5,.1)--++(0,-1)
            ++(.4,0)node[anchor=center](cdots){$\cdots$}
            ($(origin3)+(1.4,.75)$)--++(0,-1)
            (omega)circle[x radius=.8, y radius=.3]
            ++(0,-.3)--($(origin3)+(.1,0)$)--++(0,-.25)
            ;
        \end{tikzpicture}
    }}
\end{equation*}
\eqref{eqfig: kform: cartan1}第2項は$V$を反対称テンソルから外す.
\begin{equation*}
    \begin{split}
        \frac{1}{k!}
        \vcenter{\hbox{
            \begin{tikzpicture}
                \coordinate(origin1)at(-.5,0);
                \draw[ultra thick](origin1)--++(1.8,0);
                \draw($(origin1)+(.3,.25)$)rectangle++(1.5,.4);
                \node at($(origin1)+(1.05,.45)$)[anchor=center]{$\omega$};
                \draw
                ($(origin1)+(.5,.25)$)--++(0,-.5)
                ++(.4,.5)--++(0,-.5)
                ++(.4,0)node[anchor=center](cdots){$\cdots$}
                ($(origin1)+(1.7,.25)$)--++(0,-.5);
                \draw
                ($(origin1)+(1.05,.45)$)circle[x radius=1.,y radius=.35]
                ($(origin1)+(.05,.4)$)--++(0,-.65)node[anchor=north, draw, rectangle]{$V$}
                ;
            \end{tikzpicture}
        }}
        &=
        \frac{1}{k!}
        \vcenter{\hbox{
            \begin{tikzpicture}
                \coordinate(origin2)at(2.5,0);
                \draw[ultra thick]($(origin2)+(.4,0)$)--++(1.4,0);
                \draw($(origin2)+(.3,.25)$)rectangle++(1.5,.4);
                \node at($(origin2)+(1.05,.45)$)[anchor=center]{$\omega$};
                \draw
                ($(origin2)+(.5,.25)$)--++(0,-.5)
                ++(.4,.5)--++(0,-.5)
                ++(.4,0)node[anchor=center](cdots){$\cdots$}
                ($(origin2)+(1.7,.25)$)--++(0,-.5);
                \draw
                ($(origin2)+(1.05,.45)$)circle[x radius=1.,y radius=.35]
                ($(origin2)+(.05,.4)$)--++(0,-.65)node[anchor=north, draw, rectangle]{$V$}
                ;
            \end{tikzpicture}
        }}
        -
        \frac{1}{k!}
        \vcenter{\hbox{
            \begin{tikzpicture}
                \coordinate(origin3)at(5.5,0);
                \draw[ultra thick]($(origin3)+(.0,0)$)--++(1.5,0);
                \draw($(origin3)+(0,.75)$)rectangle++(1.5,.4);
                \node at($(origin3)+(.75,.95)$)[anchor=center](omega){$\omega$};
                \draw
                ($(origin3)+(.1,.75)$)--++(0,-.1)node[anchor=north,draw,rectangle]{$V$}
                ++(.5,.1)--++(0,-1)
                ++(.5,1)--++(0,-1)
                ++(.4,0)node[anchor=center](cdots){$\cdots$}
                (omega)circle[x radius=.8, y radius=.3]
                ++(0,-.3)--($(origin3)+(.1,0)$)--++(0,-.25)
                ;
            \end{tikzpicture}
        }}
        +
        \frac{1}{k!}
        \vcenter{\hbox{
            \begin{tikzpicture}
                \coordinate(origin4)at(8.,0);
                \draw[ultra thick]($(origin4)+(.0,0)$)--++(1.5,0);
                \draw($(origin4)+(0,.75)$)rectangle++(1.5,.4);
                \node at($(origin4)+(.75,.95)$)[anchor=center](omega){$\omega$};
                \draw
                ($(origin4)+(.1,.75)$)--++(.5,-.75)--++(0,-.25)
                ++(0,1)--++(0,-.1)node[anchor=north,draw,rectangle]{$V$}
                ++(.5,.1)--++(0,-1)
                ++(.4,-.)node[anchor=center](cdots){$\cdots$}
                (omega)circle[x radius=.8, y radius=.3]
                ++(-.8,0)--++(-.1,0)--($(origin4)+(.1,0)$)--++(0,-.25)
                ;
            \end{tikzpicture}
        }}
        -
        \cdots
        \\
        &=
        \frac{1}{k!}
        \vcenter{\hbox{
            \begin{tikzpicture}
                \coordinate(origin5)at(2.5,-2.);
                \draw[ultra thick]($(origin5)+(.4,0)$)--++(1.4,0);
                \draw($(origin5)+(.3,.25)$)rectangle++(1.5,.4);
                \node at($(origin5)+(1.05,.45)$)[anchor=center]{$\omega$};
                \draw
                ($(origin5)+(.5,.25)$)--++(0,-.5)
                ++(.4,.5)--++(0,-.5)
                ++(.4,0)node[anchor=center](cdots){$\cdots$}
                ($(origin5)+(1.7,.25)$)--++(0,-.5);
                \draw
                ($(origin5)+(1.05,.45)$)circle[x radius=1.,y radius=.35]
                ($(origin5)+(.05,.4)$)--++(0,-.65)node[anchor=north, draw, rectangle]{$V$}
                ;
            \end{tikzpicture}
        }}
        -
        \frac{k}{k!}
        \vcenter{\hbox{
            \begin{tikzpicture}
                \coordinate(origin6)at(5.5,-2.);
                \draw[ultra thick]($(origin6)+(.0,0)$)--++(1.5,0);
                \draw($(origin6)+(0,.75)$)rectangle++(1.5,.4);
                \node at($(origin6)+(.75,.95)$)[anchor=center](omega){$\omega$};
                \draw
                ($(origin6)+(.1,.75)$)--++(0,-.1)node[anchor=north,draw,rectangle]{$V$}
                ++(.5,.1)--++(0,-1)
                ++(.5,1)--++(0,-1)
                ++(.4,0)node[anchor=center](cdots){$\cdots$}
                (omega)circle[x radius=.8, y radius=.3]
                ++(0,-.3)--($(origin6)+(.1,0)$)--++(0,-.25)
                ;
            \end{tikzpicture}
        }}
    \end{split}
\end{equation*}
1段目右辺第2項以降で, 符号が正の項は偶置換で, 負の項は奇置換で全て第2項になるので, $V$が微分に繋がっている初項を除き右辺は全て同じ項である.
第2項以降は$V$が$\omega$に繋がる位置を考慮すると全部で$k$項あるので, 2段目を得る.

従って\eqref{eqfig: kform: cartan1}は以下の形に等しい.
\begin{equation}
    \label{eqfig: kform: cartan4}
    (\dd{\iota_V}+\iota_V\dd{})
    \frac{1}{k!}
    \vcenter{\hbox{
        \begin{tikzpicture}
            \coordinate(origin1)at(-.3,0);
            \draw[ultra thick](origin1)--++(1.5,0);
            \draw($(origin1)+(0,.25)$)rectangle++(1.5,.4);
            \node at($(origin1)+(.75,.45)$)[anchor=center]{$\omega$};
            \draw
            ($(origin1)+(.1,.25)$)--++(0,-.5)node[anchor=north](mu1){$\mu_1$}
            ++(.4,.5)--++(0,-.5)node[anchor=north](mu2){$\mu_2$}
            ++(.4,0)node[anchor=center](cdots){$\cdots$}
            ($(origin1)+(1.4,.25)$)--++(0,-.5)node[anchor=north](muk){$\mu_k$};
        \end{tikzpicture}
    }}
    =
    \frac{k}{k!}
    \vcenter{\hbox{
        \begin{tikzpicture}
            \coordinate(origin2)at(3,0);
            \draw[ultra thick]($(origin2)+(.0,0)$)--++(1.5,0);
            \draw($(origin2)+(0,.75)$)rectangle++(1.5,.4);
            \node at($(origin2)+(.75,.95)$)[anchor=center]{$\omega$};
            \draw
            ($(origin2)+(.1,.75)$)--++(0,-.1)node[anchor=north,draw,rectangle](V){$V$}
            ++(.5,.1)--++(0,-1)
            ++(.4,0)node[anchor=center](cdots){$\cdots$}
            ($(origin2)+(1.4,.75)$)--++(0,-1)
            (V)circle[radius=.3]
            ++(0,-.3)--($(origin2)+(.1,0)$)--++(0,-.25)
            ;
        \end{tikzpicture}
    }}
    +
    \frac{1}{k!}
    \vcenter{\hbox{
        \begin{tikzpicture}
            \coordinate(origin3)at(5.5,0);
            \draw[ultra thick]($(origin3)+(.4,0)$)--++(1.4,0);
            \draw($(origin3)+(.3,.25)$)rectangle++(1.5,.4);
            \node at($(origin3)+(1.05,.45)$)[anchor=center]{$\omega$};
            \draw
            ($(origin3)+(.5,.25)$)--++(0,-.5)
            ++(.4,.5)--++(0,-.5)
            ++(.4,0)node[anchor=center](cdots){$\cdots$}
            ($(origin3)+(1.7,.25)$)--++(0,-.5);
            \draw
            ($(origin3)+(1.05,.45)$)circle[x radius=1.,y radius=.35]
            ($(origin3)+(.05,.4)$)--++(0,-.65)node[anchor=north, draw, rectangle]{$V$}
            ;
        \end{tikzpicture}
    }}
\end{equation}
さらにこの右辺第1項は$\omega_{\mu_1\cdots\mu_k}$の添字に対する反対称性から以下のように展開できる.
\begin{equation*}
    \begin{split}
        \frac{k}{k!}
        \vcenter{\hbox{
            \begin{tikzpicture}
                \coordinate(origin2)at(3,0);
                \draw[ultra thick]($(origin2)+(.0,0)$)--++(1.5,0);
                \draw($(origin2)+(0,.75)$)rectangle++(1.5,.4);
                \node at($(origin2)+(.75,.95)$)[anchor=center]{$\omega$};
                \draw
                ($(origin2)+(.1,.75)$)--++(0,-.1)node[anchor=north,draw,rectangle](V){$V$}
                ++(.5,.1)--++(0,-1)
                ++(.4,0)node[anchor=center](cdots){$\cdots$}
                ($(origin2)+(1.4,.75)$)--++(0,-1)
                (V)circle[radius=.3]
                ++(0,-.3)--($(origin2)+(.1,0)$)--++(0,-.25)
                ;
            \end{tikzpicture}
        }}
        &=
        \frac{1}{k!}
        \vcenter{\hbox{
            \begin{tikzpicture}
                \coordinate(origin2)at(3,0);
                \draw[ultra thick]($(origin2)+(.0,0)$)--++(1.5,0);
                \draw($(origin2)+(0,.75)$)rectangle++(1.5,.4);
                \node at($(origin2)+(.75,.95)$)[anchor=center]{$\omega$};
                \draw
                ($(origin2)+(.1,.75)$)--++(0,-.1)node[anchor=north,draw,rectangle](V){$V$}
                ++(.5,.1)--++(0,-1)
                ++(.4,0)node[anchor=center](cdots){$\cdots$}
                ($(origin2)+(1.4,.75)$)--++(0,-1)
                (V)circle[radius=.3]
                ++(0,-.3)--($(origin2)+(.1,0)$)--++(0,-.25)
                ;
            \end{tikzpicture}
        }}
        -
        \frac{1}{k!}
        \vcenter{\hbox{
            \begin{tikzpicture}
                \coordinate(origin3)at(5.,0);
                \draw[ultra thick]($(origin3)+(.0,0)$)--++(1.5,0);
                \draw($(origin3)+(0,.75)$)rectangle++(1.5,.4);
                \node at($(origin3)+(.75,.95)$)[anchor=center]{$\omega$};
                \draw
                ($(origin3)+(.1,.75)$)--++(0,-.6)--($(origin3)+(.6,0)$)--++(0,-.25)
                ++(.,1)--++(0,-.1)node[anchor=north,draw,rectangle](V){$V$}
                ++(.5,-.9)node[anchor=center](cdots){$\cdots$}
                ($(origin3)+(1.4,.75)$)--++(0,-1)
                (V)circle[radius=.3]
                ++(0,-.3)--($(origin3)+(.1,0)$)--++(0,-.25)
                ;
            \end{tikzpicture}
        }}
        +
        \cdots
        \\
        &=
        \frac{1}{k!}
        \vcenter{\hbox{
            \begin{tikzpicture}
                \coordinate(origin2)at(3,0);
                \draw[ultra thick]($(origin2)+(.0,0)$)--++(1.5,0);
                \draw($(origin2)+(0,.75)$)rectangle++(1.5,.4);
                \node at($(origin2)+(.75,.95)$)[anchor=center]{$\omega$};
                \draw
                ($(origin2)+(.1,.75)$)--++(0,-.1)node[anchor=north,draw,rectangle](V){$V$}
                ++(.5,.1)--++(0,-1)
                ++(.4,0)node[anchor=center](cdots){$\cdots$}
                ($(origin2)+(1.4,.75)$)--++(0,-1)
                (V)circle[radius=.3]
                ++(0,-.3)--($(origin2)+(.1,0)$)--++(0,-.25)
                ;
            \end{tikzpicture}
        }}
        +
        \vcenter{\hbox{
            \begin{tikzpicture}
                \coordinate(origin5)at(5.,-1.5);
                \draw[ultra thick]($(origin5)+(.0,0)$)--++(1.5,0);
                \draw($(origin5)+(0,.75)$)rectangle++(1.5,.4);
                \node at($(origin5)+(.75,.95)$)[anchor=center]{$\omega$};
                \draw
                ($(origin5)+(.1,.75)$)--++(0,-1)
                ++(.5,1)--++(0,-.1)node[anchor=north,draw,rectangle](V){$V$}
                ++(.4,-.75)node[anchor=center](cdots){$\cdots$}
                ($(origin5)+(1.4,.75)$)--++(0,-1)
                (V)circle[radius=.3]
                ++(0,-.3)--($(origin5)+(.6,0)$)--++(0,-.25)
                ;
            \end{tikzpicture}
        }}
        +
        \cdots
    \end{split}
\end{equation*}
\eqref{eqfig: kform: cartan4}に戻すと, これはk-formのLie微分に等しい.
よってCartanの公式を得る.
\begin{equation}
    \label{eq: k-form: Cartan formula result}
    (\dd{\iota_V}+\iota_V\dd{})\omega=\mathcal{L}_V\omega
\end{equation}



\subsection{Riemann多様体上の微分形式}
\label{sec: k-form: differential form on riemann manifold}

以下では計量$g=g_{\mu\nu}\dd{x}^\mu\otimes\dd{x}^\nu$が入った$n$次元Riemann多様体または擬Riemann多様体を扱う.
$g$を計量テンソルの絶対値を$g=\det(g_{\mu\nu})$とする.
擬Riemann多様体まで含めると, 必ずしも正とは限らない.
反変Levi-Civitaテンソルは
\eqref{eq: k-form: contravariant Levi-Civita}
によって定義する.


\subsubsection{Hodge star}
\label{sec: k-form: Hodge star}

微分形式を$k$-formから$(n-k)$-formへ移すHodge star $*$の作用は
\begin{equation}
    \label{eq: k-form: Hodge star def in Einstein contraction}
    \begin{split}
        &
        \ast\qty(
            \frac{1}{k!}
            \omega_{\mu_1\cdots\mu_k}
            \dd{x}^{\mu_1}
            \wedge\cdots\wedge
            \dd{x}^{\mu_k}
        )
        \\
        &\qquad
        =
        \frac{\sqrt{|g|}}{k!(n-k)!}
        g^{\mu_1\nu_1}\cdots g^{\mu_k\nu_k}
        \omega_{\nu_1\cdots\nu_k}
        \epsilon_{\mu_1\cdots\mu_k\mu_{k+1}\cdots\mu_n}
        \dd{x}^{\mu_{k+1}}
        \wedge\cdots\wedge
        \dd{x}^{\mu_n}
        \\
        &\qquad
        =
        \frac{\sqrt{|g|}}{k!(n-k)!}
        g^{\mu_1\nu_1}\cdots g^{\mu_k\nu_k}
        \omega_{\nu_1\cdots\nu_k}
        \epsilon_{\mu_1\cdots\mu_k\mu_{k+1}\cdots\mu_n}
        \epsilon^{\mu_{k+1}\cdots\mu_{n}}_{\rho_{k+1}\cdots\rho_{n}}
        \dd{x}^{\rho_{k+1}}
        \otimes\cdots\otimes
        \dd{x}^{\rho_n}
    \end{split}
\end{equation}
と表される.
成分は以下のように描ける.
\begin{equation*}
    \ast\qty(
        \frac{1}{k!}
        \vcenter{\hbox{\begin{tikzpicture}
    \coordinate(origin1)at(0,0);
    \draw[ultra thick](origin1)--++(1.5,0);
    \draw($(origin1)+(0,.25)$)rectangle++(1.5,.4);
    \node at($(origin1)+(.75,.45)$)[anchor=center]{$\omega$};
    \draw
    ($(origin1)+(.1,.25)$)--++(0,-.5)node[anchor=north](mu1){$\mu_1$}
    ++(.4,.5)--++(0,-.5)node[anchor=north](mu2){$\mu_2$}
    ++(.4,0)node[anchor=center](cdots){$\cdots$}
    ($(origin1)+(1.4,.25)$)--++(0,-.5)node[anchor=north](muk){$\mu_k$};
\end{tikzpicture}
}}
    )
    =
    \frac{\sqrt{|g|}}{k!(n-k)!}
    \vcenter{\hbox{
        \begin{tikzpicture}
            \coordinate(origin)at(0,0);
            \draw[ultra thick]($(origin)+(1.8,0)$)--++(1.2,0);
            \draw($(origin)+(0,.25)$)rectangle++(1.5,.4);
            \draw
                ($(origin)+(.1,.25)$)--++(0,-.5)node[anchor=north](mu1){}
                ++(.4,.5)--++(0,-.5)node[anchor=north](mu2){}
                ++(.5,0)node[anchor=center](cdots){$\cdots$}
                ($(origin)+(1.4,.25)$)--++(0,-.5)node[anchor=north](muk){}
                (mu1.north)
                arc(-180:0:.1)node[anchor=north](mu1up){}
                (mu2.north)
                arc(-180:0:.1)node[anchor=north](mu2up){}
                (muk.north)
                arc(-180:0:.1)node[anchor=north](mu3up){}
                ($(origin)+(2,1.)$)--++(0,-1.25)node[anchor=north]{$\mu_{k+1}$}
                ++(.5,.5)node[anchor=center]{$\cdots$}
                ($(origin)+(2.9,1.)$)--++(0,-1.25)node[anchor=north]{$\mu_n$}
            ;
            \draw[preaction={draw,white,line width=3pt}]
                (mu1up.north)--++(0,1.25)
                (mu2up.north)--++(0,1.25)
                (mu3up.north)--++(0,1.25)
            ;
            \draw[ultra thick]($(origin)+(0,1)$)--++(3,0);
        \end{tikzpicture}
    }}
    =
    \frac{\sqrt{|g|}}{k!}
    \vcenter{\hbox{
        \begin{tikzpicture}
            \coordinate(origin)at(0,0);
            \draw($(origin)+(0,.25)$)rectangle++(1.5,.4);
            \draw
                ($(origin)+(.1,.25)$)--++(0,-.5)node[anchor=north](mu1){}
                ++(.4,.5)--++(0,-.5)node[anchor=north](mu2){}
                ++(.5,0)node[anchor=center](cdots){$\cdots$}
                ($(origin)+(1.4,.25)$)--++(0,-.5)node[anchor=north](muk){}
                (mu1.north)
                arc(-180:0:.1)node[anchor=north](mu1up){}
                (mu2.north)
                arc(-180:0:.1)node[anchor=north](mu2up){}
                (muk.north)
                arc(-180:0:.1)node[anchor=north](mu3up){}
                ($(origin)+(2,1.)$)--++(0,-1.25)node[anchor=north]{$\mu_{k+1}$}
                ++(.5,.5)node[anchor=center]{$\cdots$}
                ($(origin)+(2.9,1.)$)--++(0,-1.25)node[anchor=north]{$\mu_n$}
            ;
            \draw[preaction={draw,white,line width=3pt}]
                (mu1up.north)--++(0,1.25)
                (mu2up.north)--++(0,1.25)
                (mu3up.north)--++(0,1.25)
            ;
            \draw[ultra thick]($(origin)+(0,1)$)--++(3,0);
        \end{tikzpicture}
    }}
\end{equation*}


\subsubsection{余微分}

\begin{equation*}
    \delta
    =
    (-1)^{nk+n+1}\frac{g}{|g|}\ast\dd\ast
\end{equation*}
の$k$-form $\omega$に対する作用は, これまでの図式から以下のように計算できる.
\begin{equation*}
    \begin{split}
        \delta\qty(
            \frac{1}{k!}
            \vcenter{\hbox{\begin{tikzpicture}
    \coordinate(origin1)at(0,0);
    \draw[ultra thick](origin1)--++(1.5,0);
    \draw($(origin1)+(0,.25)$)rectangle++(1.5,.4);
    \node at($(origin1)+(.75,.45)$)[anchor=center]{$\omega$};
    \draw
    ($(origin1)+(.1,.25)$)--++(0,-.5)node[anchor=north](mu1){$\mu_1$}
    ++(.4,.5)--++(0,-.5)node[anchor=north](mu2){$\mu_2$}
    ++(.4,0)node[anchor=center](cdots){$\cdots$}
    ($(origin1)+(1.4,.25)$)--++(0,-.5)node[anchor=north](muk){$\mu_k$};
\end{tikzpicture}
}}
        )
        &=
        (-1)^{nk+n+1}
        \frac{g}{|g|}
        \frac{1}{k!}
        \ast\dd
        \qty(
            \frac{\sqrt{|g|}}{k!(n-k)!}
            \vcenter{\hbox{
                \begin{tikzpicture}
                    \coordinate(origin)at(0,0);
                    \draw[ultra thick]($(origin)+(1.8,0)$)--++(1.2,0);
                    \draw($(origin)+(0,.25)$)rectangle++(1.5,.4);
                    \draw
                        ($(origin)+(.1,.25)$)--++(0,-.5)node[anchor=north](mu1){}
                        ++(.4,.5)--++(0,-.5)node[anchor=north](mu2){}
                        ++(.5,0)node[anchor=center](cdots){$\cdots$}
                        ($(origin)+(1.4,.25)$)--++(0,-.5)node[anchor=north](muk){}
                        (mu1.north)
                        arc(-180:0:.1)node[anchor=north](mu1up){}
                        (mu2.north)
                        arc(-180:0:.1)node[anchor=north](mu2up){}
                        (muk.north)
                        arc(-180:0:.1)node[anchor=north](mu3up){}
                        ($(origin)+(2,1.)$)--++(0,-1.25)node[anchor=north]{}
                        ++(.5,.5)node[anchor=center]{$\cdots$}
                        ($(origin)+(2.9,1.)$)--++(0,-1.25)node[anchor=north]{}
                    ;
                    \draw[preaction={draw,white,line width=3pt}]
                        (mu1up.north)--++(0,1.25)
                        (mu2up.north)--++(0,1.25)
                        (mu3up.north)--++(0,1.25)
                    ;
                    \draw[ultra thick]($(origin)+(0,1)$)--++(3,0);
                \end{tikzpicture}
            }}
        )
        \\
        &=
        \frac{(-1)^{nk+n+1}}{(k!)^2(n-k)!}
        \frac{g}{|g|}
        \ast\qty(
            \vcenter{\hbox{
                \begin{tikzpicture}
                    \coordinate(origin)at(0,0);
                    \draw[ultra thick]($(origin)+(1.4,-.6)$)--++(1.6,0);
                    \draw($(origin)+(0,.25)$)rectangle++(1.5,.4);
                    \draw
                        ($(origin)+(.1,.25)$)--++(0,-.3)node[anchor=north](mu1){}
                        ++(.4,.3)--++(0,-.3)node[anchor=north](mu2){}
                        ++(.5,0)node[anchor=center](cdots){$\cdots$}
                        ($(origin)+(1.4,.25)$)--++(0,-.3)node[anchor=north](muk){}
                        (mu1.north)
                        arc(-180:0:.1)node[anchor=north](mu1up){}
                        (mu2.north)
                        arc(-180:0:.1)node[anchor=north](mu2up){}
                        (muk.north)
                        arc(-180:0:.1)node[anchor=north](mu3up){}
                        ($(origin)+(2,1.)$)--++(0,-1.8)node[anchor=north]{}
                        ++(.5,.5)node[anchor=center]{$\cdots$}
                        ($(origin)+(2.9,1.)$)--++(0,-1.8)node[anchor=north]{}
                    ;
                    \node at($(mu1up.north)+(-.7,0)$)[anchor=south]{$\sqrt{|g|}$};
                    \draw
                        (.6,.2)circle(1.3 and .6)
                        ++(-45:1.3 and .6)--++(0,-.6)
                        node[anchor=north](muk){}
                    ;
                    \draw[preaction={draw,white,line width=3pt}]
                        (mu1up.north)--++(0,1.05)
                        (mu2up.north)--++(0,1.05)
                        (mu3up.north)--++(0,1.05)
                    ;
                    \draw[ultra thick]($(origin)+(0,1)$)--++(3,0);
                \end{tikzpicture}
            }}
        )
        \\
        &=
        \frac{(-1)^{nk+n+1}}{(k!)^2(n-k)!}
        \frac{g}{|g|}
        \frac{\sqrt{|g|}}{(k-1)!}
        \vcenter{\hbox{
            \begin{tikzpicture}
                \coordinate(origin)at(0,0);
                \draw[ultra thick]($(origin)+(1.4,-.6)$)--++(1.6,0);
                \draw($(origin)+(0,.25)$)rectangle++(1.5,.4);
                \draw
                    ($(origin)+(.1,.25)$)--++(0,-.3)node[anchor=north](mu1){}
                    ++(.4,.3)--++(0,-.3)node[anchor=north](mu2){}
                    ++(.5,0)node[anchor=center](cdots){$\cdots$}
                    ($(origin)+(1.4,.25)$)--++(0,-.3)node[anchor=north](muk){}
                    (mu1.north)
                    arc(-180:0:.1)node[anchor=north](mu1up){}
                    (mu2.north)
                    arc(-180:0:.1)node[anchor=north](mu2up){}
                    (muk.north)
                    arc(-180:0:.1)node[anchor=north](mu3up){}
                    ($(origin)+(2,1.)$)--++(0,-1.8)node[anchor=north](muk+1){}
                    ++(.5,.5)node[anchor=center]{$\cdots$}
                    ($(origin)+(2.9,1.)$)--++(0,-1.8)node[anchor=north](mun){}
                ;
                \node at($(mu1up.north)+(-.7,0)$)[anchor=south]{$\sqrt{|g|}$};
                \draw[preaction={draw,white,line width=3pt}]
                    (mu1up.north)--++(0,1.05)
                    (mu2up.north)--++(0,1.05)
                    (mu3up.north)--++(0,1.05)
                ;
                \draw
                    (.6,.2)circle(1.3 and .6)
                    ++(-45:1.3 and .6)--++(0,-.6)
                    node[anchor=north](muk){}
                ;
                \draw[ultra thick]($(origin)+(0,1)$)--++(3,0);
                \draw[preaction={draw,white,line width=3pt}]
                    (muk.north)arc(-180:0:.1)
                    --++(0,2.25)
                    (muk+1.north)arc(-180:0:.1)
                    --++(0,2.25)
                    (mun.north)arc(-180:0:.1)
                    --++(0,2.25)
                    node[anchor=center](nun){}
                ;
                \draw[ultra thick]($(muk.north)+(-.1,2.25)$)--++(3,0);
                \draw
                    ($(nun)+(.3,0)$)--++(0,-3)
                    ($(nun)+(1.2,0)$)--++(0,-3)
                    ++(-.45,1)node[anchor=center]{$\cdots$}
                ;
                \draw[ultra thick]($(mun.south)+(.3,-.2)$)--++(1.3,0);
            \end{tikzpicture}
        }}
    \end{split}
\end{equation*}
微分が$\omega$から直接つながる$g^{ij}$や係数$\sqrt{|g|}$にも作用していることに注意.


\subsection{Riemann多様体上の微分形式の公式}
\label{sec: kform: formulae of k-form on Riemann manifold}

\subsubsection{Hodge starの2回作用}

$M$上の$k$-form $\omega$について,
\begin{equation*}
    \ast\ast\omega
    =
    (-1)^{k(n-k)}\frac{g}{|g|}\omega
\end{equation*}
であることは
\begin{equation*}
    \begin{split}
        \ast\ast\qty(
            \frac{1}{k!}
            \vcenter{\hbox{\begin{tikzpicture}
    \coordinate(origin1)at(0,0);
    \draw[ultra thick](origin1)--++(1.5,0);
    \draw($(origin1)+(0,.25)$)rectangle++(1.5,.4);
    \node at($(origin1)+(.75,.45)$)[anchor=center]{$\omega$};
    \draw
    ($(origin1)+(.1,.25)$)--++(0,-.5)node[anchor=north](mu1){$\mu_1$}
    ++(.4,.5)--++(0,-.5)node[anchor=north](mu2){$\mu_2$}
    ++(.4,0)node[anchor=center](cdots){$\cdots$}
    ($(origin1)+(1.4,.25)$)--++(0,-.5)node[anchor=north](muk){$\mu_k$};
\end{tikzpicture}
}}
        )
        &=
        \frac{(\sqrt{|g|})^2}{k!(n-k)!}
        \vcenter{\hbox{
            \begin{tikzpicture}
                \coordinate(origin)at(0,0);
                \draw($(origin)+(0,.25)$)rectangle++(1.5,.4);
                \draw[ultra thick]($(origin)+(0,1)$)--++(3,0);
                \draw[ultra thick]($(origin)+(2,2)$)--++(3.2,0);
                \draw
                    ($(origin)+(.1,.25)$)--++(0,-.5)node[anchor=north](mu1){}
                    ++(.4,.5)--++(0,-.5)node[anchor=north](mu2){}
                    ++(.5,0)node[anchor=center](cdots){$\cdots$}
                    ($(origin)+(1.4,.25)$)--++(0,-.5)node[anchor=north](muk){}
                    (mu1.north)
                    arc(-180:0:.1)node[anchor=north](mu1up){}
                    (mu2.north)
                    arc(-180:0:.1)node[anchor=north](mu2up){}
                    (muk.north)
                    arc(-180:0:.1)node[anchor=north](mu3up){}
                    ($(origin)+(2,1.)$)--++(0,-1.25)node[anchor=north](muk1){}
                    ++(.5,.5)node[anchor=center]{$\cdots$}
                    ($(origin)+(2.9,1.)$)--++(0,-1.25)node[anchor=north](mun){}
                    (muk1.north)arc(-180:0:.1)node[anchor=north](muk1up){}
                    (mun.north)arc(-180:0:.1)node[anchor=north](munup){}
                    ($(origin)+(3.7,2.)$)--++(0,-.5)
                    node[anchor=north]{$\mu_1$}
                    ++(.4,.5)--++(0,-.5)
                    node[anchor=north]{$\mu_2$}
                    ++(.5,.2)node[anchor=center]{$\cdots$}
                    ($(origin)+(5,2)$)--++(0,-.5)
                    node[anchor=north]{$\mu_k$}
                ;
                \draw[preaction={draw,white,line width=3pt}]
                    (mu1up.north)--++(0,1.2)
                    (mu2up.north)--++(0,1.2)
                    (mu3up.north)--++(0,1.2)
                    (muk1up.north)--++(0,2.2)
                    (munup.north)--++(0,2.2)
                ;
            \end{tikzpicture}
        }}
        \\
        &=
        (-1)^{k(n-k)}
        \frac{|g|}{k!(n-k)!}
        \vcenter{\hbox{
            \begin{tikzpicture}
                \coordinate(origin)at(0,0);
                \draw($(origin)+(0,.25)$)rectangle++(1.5,.4);
                \draw[ultra thick]($(origin)+(0,1)$)--++(3,0);
                \draw[ultra thick]($(origin)+(0,2)$)--++(3.2,0);
                \draw
                    ($(origin)+(.1,.25)$)--++(0,-.5)node[anchor=north](mu1){}
                    ++(.4,.5)--++(0,-.5)node[anchor=north](mu2){}
                    ++(.5,0)node[anchor=center](cdots){$\cdots$}
                    ($(origin)+(1.4,.25)$)--++(0,-.5)node[anchor=north](muk){}
                    (mu1.north)
                    arc(-180:0:.1)node[anchor=north](mu1up){}
                    (mu2.north)
                    arc(-180:0:.1)node[anchor=north](mu2up){}
                    (muk.north)
                    arc(-180:0:.1)node[anchor=north](mu3up){}
                    ($(origin)+(2,1.)$)--++(0,-1.25)node[anchor=north](muk1){}
                    ++(.5,.5)node[anchor=center]{$\cdots$}
                    ($(origin)+(2.9,1.)$)--++(0,-1.25)node[anchor=north](mun){}
                    (muk1.north)arc(-180:0:.1)node[anchor=north](muk1up){}
                    (mun.north)arc(-180:0:.1)node[anchor=north](munup){}
                    ($(origin)+(.2,2.)$)--++(0,-.5)
                    node[anchor=north]{$\mu_1$}
                    ++(.4,.5)--++(0,-.5)
                    node[anchor=north]{$\mu_2$}
                    ++(.5,.2)node[anchor=center]{$\cdots$}
                    ($(origin)+(1.5,2)$)--++(0,-.5)
                    node[anchor=north]{$\mu_k$}
                ;
                \draw[preaction={draw,white,line width=3pt}]
                    (mu1up.north)--++(0,1.2)
                    (mu2up.north)--++(0,1.2)
                    (mu3up.north)--++(0,1.2)
                    (muk1up.north)--++(0,2.2)
                    (munup.north)--++(0,2.2)
                ;
            \end{tikzpicture}
        }}
        \\
        &=
        (-1)^{k(n-k)}
        \frac{|g|}{k!(n-k)!}g^{-1}
        \vcenter{\hbox{
            \begin{tikzpicture}
                \coordinate(origin)at(0,0);
                \draw($(origin)+(0,.25)$)rectangle++(1.5,.4);
                \node at($(origin)+(.75,.425)$)[anchor=center]{$\omega$};
                \draw[ultra thick]($(origin)+(0,-.25)$)--++(3,0);
                \draw[ultra thick]($(origin)+(0,-.75)$)--++(3.,0);
                \draw
                    ($(origin)+(.1,.25)$)--++(0,-.5)node[anchor=north](mu1){}
                    ++(.4,.5)--++(0,-.5)node[anchor=north](mu2){}
                    ++(.5,0)node[anchor=center](cdots){$\cdots$}
                    ($(origin)+(1.4,.25)$)--++(0,-.5)node[anchor=north](muk){}
                    ($(origin)+(2,-.25)$)--++(0,-.5)node[anchor=north](muk1){}
                    ++(.5,.25)node[anchor=center]{$\cdots$}
                    ($(origin)+(2.9,-.25)$)--++(0,-.5)node[anchor=north](mun){}
                    ($(origin)+(.1,-.75)$)--++(0,-.5)
                    node[anchor=north]{$\mu_1$}
                    ++(.4,.5)--++(0,-.5)
                    node[anchor=north]{$\mu_2$}
                    ++(.5,.2)node[anchor=center]{$\cdots$}
                    ($(origin)+(1.4,-.75)$)--++(0,-.5)
                    node[anchor=north]{$\mu_k$}
                ;
            \end{tikzpicture}
        }}
        \qquad\because\eqref{eq: levicivita: covar-contravar change of d-component Levi-Civita}
        \\
        &=
        (-1)^{k(n-k)}\frac{1}{k!(n-k)!}
        \frac{|g|}{g}(n-k)!
        \vcenter{\hbox{\begin{tikzpicture}
    \coordinate(origin1)at(0,0);
    \draw[ultra thick](origin1)--++(1.5,0);
    \draw($(origin1)+(0,.25)$)rectangle++(1.5,.4);
    \node at($(origin1)+(.75,.45)$)[anchor=center]{$\omega$};
    \draw
    ($(origin1)+(.1,.25)$)--++(0,-.5)node[anchor=north](mu1){$\mu_1$}
    ++(.4,.5)--++(0,-.5)node[anchor=north](mu2){$\mu_2$}
    ++(.4,0)node[anchor=center](cdots){$\cdots$}
    ($(origin1)+(1.4,.25)$)--++(0,-.5)node[anchor=north](muk){$\mu_k$};
\end{tikzpicture}
}}
        \qquad\because\eqref{eq: levicivita: contraction of levi-civita in general dim}
    \end{split}
\end{equation*}
からわかる.


\subsubsection{微分形式の外積}

$k$-form
\begin{equation*}
    \xi
    =
    \frac{1}{k!}\xi_{\mu_1\cdots\mu_k}
    \dd{x}^{\mu_1}
    \wedge\cdots\wedge
    \dd{x}^{\mu_k}
    ,\qquad
    \eta
    =
    \frac{1}{k!}\eta_{\mu_1\cdots\mu_k}
    \dd{x}^{\mu_1}
    \wedge\cdots\wedge
    \dd{x}^{\mu_k}
\end{equation*}
の外積が
\begin{equation*}
    \ev{\xi,\eta}
    =
    \xi\wedge\ast\eta
    =
    \frac{1}{k!}
    \xi_{\mu_1\cdots\mu_k}\eta^{\mu_1\cdots\mu_k}
    \sqrt{|g|}\dd{x}^1
    \wedge\cdots\wedge\dd{x}^n
    =
    \ev{\eta,\xi}
\end{equation*}
であることは,
\begin{equation*}
    \begin{split}
        \frac{1}{k!}
        \frac{\sqrt{|g|}}{k!(n-k)!}
        \vcenter{\hbox{
            \begin{tikzpicture}
                \coordinate(origin)at(0,0);
                \coordinate(origin2)at(0,-.6);
                \draw($(origin)+(0,.25)$)rectangle++(1.5,.4);
                \draw(origin2)rectangle++(1.5,-.4);
                \node at($(origin)+(0,.25)+(.9,.2)$)[anchor=center]{$\eta$};
                \node at($(origin2)+(.75,-.2)$)[anchor=center]{$\xi$};
                \draw[ultra thick]($(origin2)+(0,-.75)$)--++(3,0);
                \draw
                    ($(origin)+(.1,.25)$)--++(0,-.5)node[anchor=north](mu1){}
                    ++(.4,.5)--++(0,-.5)node[anchor=north](mu2){}
                    ++(.5,0)node[anchor=center](cdots){$\cdots$}
                    ($(origin)+(1.4,.25)$)--++(0,-.5)node[anchor=north](muk){}
                    (mu1.north)
                    arc(-180:0:.1)node[anchor=north](mu1up){}
                    (mu2.north)
                    arc(-180:0:.1)node[anchor=north](mu2up){}
                    (muk.north)
                    arc(-180:0:.1)node[anchor=north](mu3up){}
                    ($(origin)+(2,1.)$)--++(0,-2.75)
                    ++(.5,1)node[anchor=center]{$\cdots$}
                    ($(origin)+(2.9,1.)$)--++(0,-2.75)
                ;
                \draw[preaction={draw,white,line width=3pt}]
                    (mu1up.north)--++(0,1.25)
                    (mu2up.north)--++(0,1.25)
                    (mu3up.north)--++(0,1.25)
                ;
                \draw[ultra thick]($(origin)+(0,1)$)--++(3,0);
                \draw
                    ($(origin2)+(.1,-.4)$)--++(0,-.75)
                    ($(origin2)+(.5,-.4)$)--++(0,-.75)
                    ($(origin2)+(1.4,-.4)$)--++(0,-.75)
                ;
            \end{tikzpicture}
        }}
        &=
        \frac{\sqrt{|g|}}{(k!)^2(n-k)!}
        \vcenter{\hbox{
            \begin{tikzpicture}
                \coordinate(origin)at(0,0);
                \coordinate(origin2)at(0,-.6);
                \draw($(origin)+(0,.25)$)rectangle++(1.5,.4);
                \draw(origin2)rectangle++(1.5,-.4);
                \node at($(origin)+(0,.25)+(.9,.2)$)[anchor=center]{$\eta$};
                \node at($(origin2)+(.75,-.2)$)[anchor=center]{$\xi$};
                \draw[ultra thick]($(origin2)+(0,-.75)$)--++(3,0);
                \draw
                    ($(origin)+(.1,.25)$)--++(0,-.5)node[anchor=north](mu1){}
                    ++(.4,.5)--++(0,-.5)node[anchor=north](mu2){}
                    ++(.5,0)node[anchor=center](cdots){$\cdots$}
                    ($(origin)+(1.4,.25)$)--++(0,-.5)node[anchor=north](muk){}
                    (mu1.north)
                    arc(-180:0:.1)node[anchor=north](mu1up){}
                    (mu2.north)
                    arc(-180:0:.1)node[anchor=north](mu2up){}
                    (muk.north)
                    arc(-180:0:.1)node[anchor=north](mu3up){}
                    ($(origin)+(2,1.)$)--++(0,-2.35)
                    ++(.5,1)node[anchor=center]{$\cdots$}
                    ($(origin)+(2.9,1.)$)--++(0,-2.35)
                ;
                \draw[preaction={draw,white,line width=3pt}]
                    (mu1up.north)--++(0,1.25)
                    (mu2up.north)--++(0,1.25)
                    (mu3up.north)--++(0,1.25)
                ;
                \draw[ultra thick]($(origin)+(0,1)$)--++(3,0);
                \draw
                    ($(origin2)+(.1,-.4)$)--++(0,-.35)
                    ($(origin2)+(.5,-.4)$)--++(0,-.35)
                    ($(origin2)+(1.4,-.4)$)--++(0,-.35)
                ;
                \draw[ultra thick]($(origin2)+(0,-.9)$)--++(3,0);
                \draw
                    ($(origin2)+(.1,-.9)$)--++(0,-.5)
                    ($(origin2)+(.5,-.9)$)--++(0,-.5)
                    ($(origin2)+(1.4,-.9)$)--++(0,-.5)
                    ($(origin2)+(2,-.9)$)--++(0,-.5)
                    ($(origin2)+(2.9,-.9)$)--++(0,-.5)
                ;
            \end{tikzpicture}
        }}
        \\
        &=
        \frac{1}{(k!)^2(n-k)!}
        (n-k)!
        \vcenter{\hbox{
            \begin{tikzpicture}
                \node at(.75,.45)[anchor=center]{$\xi$};
                \draw(-.2,-.25)rectangle++(1.5,-.4);
                \node at(.5,-.45)[anchor=center]{$\eta$};
                \draw
                    (-.1,-.65)--++(0,-.25)
                    arc(-180:0:.1)
                    node[anchor=south](mu1){}
                    (1.2,-.65)--++(0,-.25)
                    arc(-180:0:.1)
                    node[anchor=south](mu2){}
                ;
                \node at($(mu1)!.5!(mu2)$)[anchor=north]{$\dots$};
                \draw[preaction={draw,white,line width=3pt}]
                    (mu1.south)--++(0,1.15)
                    (mu2.south)--++(0,1.15)
                ;
                \draw[ultra thick](0,0)--++(1.5,0);
                \draw(0,.25)rectangle++(1.5,.4);
            \end{tikzpicture}
        }}
        \sqrt{|g|}
        \vcenter{\hbox{
            \begin{tikzpicture}
                \draw[ultra thick](0,0)--++(3,0);
                \draw(.1,0)--++(0,-.5);
                \draw(.5,0)--++(0,-.5);
                \draw(1.4,0)--++(0,-.5);
                \draw(2,0)--++(0,-.5);
                \draw(2.9,0)--++(0,-.5);
            \end{tikzpicture}
        }}
        \\
        &=
        \frac{1}{k!}
        \vcenter{\hbox{
            \begin{tikzpicture}
                \node at(.75,.45)[anchor=center]{$\xi$};
                \draw(-.2,-.25)rectangle++(1.5,-.4);
                \node at(.5,-.45)[anchor=center]{$\eta$};
                \draw
                    (-.1,-.65)--++(0,-.25)
                    arc(-180:0:.1)
                    node[anchor=south](mu1){}
                    (1.2,-.65)--++(0,-.25)
                    arc(-180:0:.1)
                    node[anchor=south](mu2){}
                ;
                \node at($(mu1)!.5!(mu2)$)[anchor=north]{$\dots$};
                \draw[preaction={draw,white,line width=3pt}]
                    (mu1.south)--++(0,1.15)
                    (mu2.south)--++(0,1.15)
                ;
                \draw(0,.25)rectangle++(1.5,.4);
            \end{tikzpicture}
        }}
        \sqrt{|g|}
        \vcenter{\hbox{
            \begin{tikzpicture}
                \draw[ultra thick](0,0)--++(3,0);
                \draw(.1,0)--++(0,-.5);
                \draw(.5,0)--++(0,-.5);
                \draw(1.4,0)--++(0,-.5);
                \draw(2,0)--++(0,-.5);
                \draw(2.9,0)--++(0,-.5);
            \end{tikzpicture}
        }}
    \end{split}
\end{equation*}
から得られる.


% \subsubsection{余微分の随伴性}

% $(k-1)$-form $\xi$と$k$-form $\eta$に対し,
% \begin{equation*}
%     \ev{\dd{\xi},\eta}
%     =
%     \ev{\xi, \delta\eta}
% \end{equation*}
% が成立する.
% 一般的にはStokesの定理から示すが, ここでは演習として直接計算によって導く.
% 左辺は簡単に書き下せて,
% \begin{equation*}
%     \ev{\dd{\xi},\eta}
%     =
%     \dd{\xi}\wedge\ast\eta
%     =
%     \frac{1}{(k-1)!}
%     \frac{\sqrt{|g|}}{k!(n-k)!}
%     \vcenter{\hbox{
%         \begin{tikzpicture}
%             \coordinate(origin)at(0,0);
%             \draw($(origin)+(0,.25)$)rectangle++(1.5,.4);
%             \node at(.85,.45)[anchor=center]{$\eta$};
%             \draw
%                 ($(origin)+(.1,.25)$)--++(0,-.5)node[anchor=north](mu1){}
%                 ++(.4,.5)--++(0,-.5)node[anchor=north](mu2){}
%                 ++(.5,0)node[anchor=center](cdots){$\cdots$}
%                 ($(origin)+(1.4,.25)$)--++(0,-.5)node[anchor=north](muk){}
%                 (mu1.north)
%                 arc(-180:0:.1)node[anchor=north](mu1up){}
%                 (mu2.north)
%                 arc(-180:0:.1)node[anchor=north](mu2up){}
%                 (muk.north)
%                 arc(-180:0:.1)node[anchor=north](mu3up){}
%                 ($(origin)+(2,1.)$)--++(0,-2.5)node[anchor=north]{}
%                 ++(.5,.5)node[anchor=center]{$\cdots$}
%                 ($(origin)+(2.9,1.)$)--++(0,-2.5)node[anchor=north]{}
%             ;
%             \draw[preaction={draw,white,line width=3pt}]
%                 (mu1up.north)--++(0,1.25)
%                 (mu2up.north)--++(0,1.25)
%                 (mu3up.north)--++(0,1.25)
%             ;
%             \draw[ultra thick]($(origin)+(0,1)$)--++(3,0);
%             \draw(.4,-.6)rectangle++(1.2,-.4);
%             \draw
%                 (1,-.8)circle(.7 and .3)
%                 node[anchor=center]{$\xi$}
%                 ++(-180:.7 and .3)-|++(-.2,-.7)
%             ;
%             \draw[ultra thick](0,-1.25)--++(3,0);
%             \draw
%                 (.5,-1)--++(0,-.5)
%                 (1.5,-1)--++(0,-.5)
%             ;
%         \end{tikzpicture}
%     }}
% \end{equation*}
% である.
% 右辺は
% \begin{equation*}
%     \frac{1}{(k-1)!}
%     \frac{g}{|g|}(-1)^{nk+n+1}
%     \frac{\sqrt{|g|}}{k!(n-k)!(k-1)!}
% \end{equation*}

\end{document}
